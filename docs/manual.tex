\documentclass[a4paper,11pt]{article}
\usepackage[utf8]{inputenc}
\usepackage{booktabs}
\usepackage{hyperref}
\usepackage{listings}  
\usepackage{tikz}
\usepackage{float}
\usepackage{color}
\usepackage{hhline}
\usepackage{amsmath}
\usepackage{natbib}
\usepackage{epstopdf}
% \usepackage{manual}
\usepackage[nottoc,numbib]{tocbibind}
\renewcommand\bibname{References}

\setcounter{tocdepth}{4}

\lstset{ 
  breaklines=true,          
}
\textwidth=450 pt
\hoffset = 0 pt
\marginparwidth = 0pt
\oddsidemargin = 0 pt

\newcommand{\HRule}{\rule{\linewidth}{0.5mm}}


\lstset{ 
  breaklines=true,          
}
\textwidth=450 pt
\hoffset = 0 pt
\marginparwidth = 0pt
\oddsidemargin = 0 pt

%Tikz
\usetikzlibrary{shapes,arrows}
\usetikzlibrary{positioning}
\tikzset{
every node/.style={rectangle,draw,minimum height=2.5em,text width=1.6cm},
molecule/.style = {fill=red!20, rounded corners},
embryo/.style = {fill=blue!20},
bkgd/.style = {fill=green!20},
fit/.style = {fill=yellow!20},
pre/.style = {fill=orange!20},
noise/.style = {fill=magenta!20}
}
\tikzstyle{level 1}=[level distance=2.5cm, sibling distance=8cm]
\tikzstyle{level 2}=[level distance=2.5cm, sibling distance=1cm]
\tikzstyle{level 3}=[level distance=2.5cm, sibling distance=1cm]

%opening
\title{PyFDP User's Guide}
\author{Alexander Bl\"a\ss le}




\begin{document}

\begin{titlepage}
\begin{center}

% \includegraphics[width=0.15\textwidth]{./logo}~\\[1cm]

% Title
\HRule \\[0.4cm]
{ \huge \bfseries PyFRAP: Software for analysis of Fluorescence Recovery After Photoconversion (FRAP) experiments\\[0.4cm] User Guide \\[0.4cm] }

\HRule \\[1.5cm]

% Author and supervisor

\large
Alexander Bl\"a\ss le and Patrick M\"uller\\[0.5cm]


Friedrich Miescher Laboratory of the Max Planck Society\\
Spemannstra\ss e 39\\
72076 T\"ubingen\\
Germany\\[0.5cm]

E-Mail: \href{mailto:alexander.blaessle@tuebingen.mpg.de}{alexander.blaessle@tuebingen.mpg.de}\\
Website: \url{http://www.fml.tuebingen.mpg.de/mueller-group/}

\vfill

\end{center}
\end{titlepage}

% \tableofcontents
% \newpage
\section{Introduction}

Fluorescence Recovery After Photobleaching (FRAP) is a microscopy-based technique for measuring protein mobility.

\noindent Here, we provide a user guide for the PyFRAP software. 
  
\section{Installation}

PyFRAP was developed as an open source graphical user interface (GUI) in Python with PyQT and SciPy in order to make it accessible and extendable across the most frequently used operating systems Ubuntu Linux, Mac OS X, and Microsoft Windows. Over the past two decades, Python has become a widely used scientific programming language and provides PyFRAP users with enormous resources and easily addable software packages \citep{Millman2011}. 

All software packages needed to run PyFRAP are freely available. PyFRAP can be installed using stand-alone executables (see Section~\ref{sec:executable}). Alternatively, users can run the PyFRAP packages from source (see Section~\ref{sec:source}), which offers the possibility to edit the PyFRAP code and to import new modules.

\subsection{Running PyFRAP using stand-alone executables}
\label{sec:executable}
Download the executable that fits your system from \href{http://people.tuebingen.mpg.de/mueller-lab/}{http://people.tuebingen.mpg.de/mueller-lab/}. This is suitable for users who want to analyze FDAP experiments and do not need to customize the PyFRAP code. A list of currently available binary files and systems on which the binaries have been tested can be found in Table \ref{tab:executables}. If there is no executable available for your system, we recommend using the Anaconda installation approach explained in Section~\ref{sec:anaconda}.
 
 \begin{table}[H]
  \small
\centering
 \begin{tabular}{c|c|c|c|c|c}
  \textbf{OS} & \textbf{Version} & \textbf{32-bit} & \textbf{64-bit} & \textbf{Executable} & \textbf{Test System} \\
\hhline{=|=|=|=|=|=}
Linux & 3.13.0-36-generic & & $\mathbf{\times}$ &  & Thinkpad x230 \\ 
\hline
 \end{tabular}
\caption{List of systems on which the currently available PyFRAP executables have been tested. The executables might also run on systems not listed here.}
\label{tab:executables}
\end{table}

\subsection{Running PyFRAP from source}
\label{sec:source}
In order to be able to edit the PyFRAP code and to import new modules, it is necessary to download and install all necessary Python packages and to run PyFRAP from source. There are two ways to do this:
 \begin{enumerate}
  \item Download and install the Anaconda Python distribution (see Section~\ref{sec:anaconda}).
  \item Download and install all Python packages manually (see Section~\ref{sec:manually}).
 \end{enumerate}
 
\subsubsection{Running PyFRAP using the Anaconda distribution}
\label{sec:anaconda}

Anaconda is a bundle of Python packages and includes most of the packages needed to run PyFRAP. To install Anaconda, follow these steps:

\begin{itemize}
 \item Go to \href{http://continuum.io/downloads}{http://continuum.io/downloads} and download the current Python 2.7.x release of Anaconda for your operating system 
 \item Launch the installer by double-clicking (Mac OS X and Windows) or 
 \begin{itemize}
 \item Open a Terminal
 \item Go to the directory containing the installer by typing
 \begin{lstlisting}[frame=single,language=bash]  
cd path/to/installer
\end{lstlisting}
and execute the installer with
 \begin{lstlisting}[frame=single,language=bash]  
./installer
\end{lstlisting}
 \end{itemize}
\item Follow the instructions of the installer
\item Download the current version of Gmsh from \url{http://geuz.org/gmsh/}. Unpack the zip file and copy it to your \textit{Programs} under Microsoft Windows or mount the .dmg file 
and copy Gmsh.app to the \textit{Applications} folder under Mac OSX.
\item Both for Microsoft Windows and Mac OSX the Gmsh executable needs to be added to the \verb+PATH+ environment variable. Instructions how to add an executable to \verb+PATH+ can be found in Sections
\ref{sec:pathosx} and \ref{pathwin} for Mac OS X and Microsoft Windows respectively. The standard paths for the Gmsh executable are  \verb+/Applications/Gmsh.app/MacOS+ for Mac OS X and \verb+????+ for Microsoft Windows.
\item Installing the pysparse package will speed FRAP simulations substantially. 

\item Launch PyFRAP by double-clicking \verb+pyfrap_app.py+ in the PyFRAP source directory (Windows) or
\begin{itemize}
 \item Open a Terminal
 \item Go to the directory containing the PyFRAP source files
 \begin{lstlisting}[frame=single,language=bash]  
cd path/to/PyFRAP
\end{lstlisting}
\item Launch PyFRAP by typing
 \begin{lstlisting}[frame=single,language=bash]  
python pyfrap_app.py
\end{lstlisting}
\end{itemize}
\end{itemize}

\subsubsection{Running PyFRAP using a manual Python installation}
\label{sec:manually}
In this section, we explain how to manually install all necessary Python packages on Linux, Mac OS X, and Windows in order to run PyFRAP. The manual installation allows for customizability as well as debugging options. The instructions provided here describe the installation process for computers that currently do not have Python installed. For computers on which Python is already installed, the installation of PyFRAP will differ from the instructions provided below. We recommend running PyFRAP using a Debian-based Linux distribution such as Ubuntu since installing Python packages is more straightforward using such operating systems.

\subsection{Manual installation under Linux}

Here we explain how to manually install and run PyFRAP on Linux operating systems. The following instructions are only suitable for Debian-based Linux distributions and have been tested on Ubuntu Linux 12.04, 13.10, and 14.04 (64-bit). Installation steps may vary between different versions and distributions of Linux (e.g. RedHat-based Linux distributions such as Fedora or Suse). .

\begin{itemize}
 \item Open a Terminal.
 \item In your Terminal, type (you will need sudo rights):
 \begin{lstlisting}[frame=single,language=bash]  
sudo apt-get install python-numpy
sudo apt-get install python-scipy
sudo apt-get install python-matplotlib
sudo apt-get install python-qt4
sudo apt-get install python-skimage
sudo apt-get install gmsh
sudo apt-get install python-fipy
\end{lstlisting}
Note: On Ubuntu versions older than 12.10, python-skimage needs to be installed from here: \url{http://neuro.debian.net/pkgs/python-skimage.html}.
 \item Go to your PyFRAP folder by typing 
\begin{lstlisting}[frame=single,language=bash]  
cd path/to/PyFRAP/
\end{lstlisting}
and launch PyFRAP by typing
\begin{lstlisting}[frame=single,language=bash]  
python pyfrp_app.py
\end{lstlisting}
If PyFRAP does not launch, open a Python Terminal and try to import all necessary packages by typing
\begin{lstlisting}[frame=single,language=Python]  
import numpy
import scipy
import matplotlib
import matplotlib.image
import PyQt4
import code
import fipy
\end{lstlisting}
This should run without any problems. If you receive an error message while importing any of these modules, please try to reinstall the package or visit the development website of the problematic package.

\end{itemize}

\subsection{Manual installation under Mac OS X}
\label{sec:macos}
Here we explain how to  manually install and run PyFDAP on Mac OS X. The following instructions have only been tested on Mac OS X 10.9.4, 10.9.5 (64-bit). Installation steps may vary between different versions of OS X.

\begin{itemize}
 \item Installing Python packages requires the C++ compiler gcc. gcc can be obtained by downloading XCode from the Apple AppStore.
 \item Launch a bash terminal under Applications $\rightarrow$ Utilities $\rightarrow$ Terminal.
 \item Type
 \begin{lstlisting}[frame=single,language=bash]  
gcc
\end{lstlisting}
You should see a popup window asking you to install Command Line Tools. Follow the instructions in the popup window.
\item Homebrew is a package manager for MacOSX that facilitates installing packages under OSX. Download homebrew by typing:
\begin{lstlisting}[frame=single,language=bash]  
ruby -e "$(curl -fsSL https://raw.github.com/Homebrew/homebrew/go/install)"
\end{lstlisting}
\item Check the homebrew installation by typing: 
\begin{lstlisting}[frame=single,language=bash]  
brew update
brew doctor
\end{lstlisting}
If the ouput returns any problems, visit the homebrew website (\url{http://brew.sh/}) for further instructions.
\item Install Python by typing into the bash terminal:
\begin{lstlisting}[frame=single,language=bash]  
brew install python
\end{lstlisting}
Note that MacOSX comes with a native Python installation. If you want to use the native Python installation, you could either install all packages 
separately by using the python package index (pip) or use homebrew to install all packages and then link them using the \verb+site+ package \url{https://docs.python.org/2/library/site.html}. 
However, we recommend using the Python installation of homebrew. 
\item Link the new homebrew installation by typing into the bash terminal
\begin{lstlisting}[frame=single,language=bash]  
brew link python
brew linkapps
\end{lstlisting}
\item Link the new python installation into .bash\_profile by launching the text editor \verb+nano+: 
\begin{lstlisting}[frame=single,language=bash]  
nano ~/.bash_profile
\end{lstlisting}
and add the following lines:
\begin{lstlisting}[frame=single,language=bash]  
PATH="/usr/local/bin:${PATH}"
export PATH
export PYTHONPATH=/usr/local/lib/python2.7/site-packages/:
\end{lstlisting}
Press \verb-Ctrl+O- and \verb-Ctrl+X- to save the new .bash\_profile and exit. Restart the terminal and type:
\begin{lstlisting}[frame=single,language=bash]  
which python 
\end{lstlisting}
The output should be
\begin{lstlisting}[frame=single,language=bash]  
/usr/local/bin/python
\end{lstlisting}
If not, ensure that you set the Python path properly or don't have a different homebrew installation prefix. If everything went right, you will now use the homebrew Python installation when you call \verb+python+ in the terminal.
\item Download and install PyQT4 and SIP by typing into the terminal
\begin{lstlisting}[frame=single,language=bash]  
brew install sip
brew install pyqt
brew linkapps
\end{lstlisting}
\item Download and install nose and Numpy by typing into the terminal
\begin{lstlisting}[frame=single,language=bash]  
pip install nose
brew install numpy 
brew link numpy
\end{lstlisting}
Sometimes numpy can also be found by typing into the terminal
\begin{lstlisting}[frame=single,language=bash]  
brew install homebrew/python/numpy
brew link numpy
\end{lstlisting}
\item Download and install Scipy by typing into the terminal
\begin{lstlisting}[frame=single,language=bash]  
pip install scipy
\end{lstlisting}
or 
\begin{lstlisting}[frame=single,language=bash]  
brew install scipy
\end{lstlisting}
\item Download and install scikit-image by typing into the terminal
\begin{lstlisting}[frame=single,language=bash]  
pip install cython
pip install scikit-image
\end{lstlisting}
\item Download and install matplotlib by typing into the terminal
\begin{lstlisting}[frame=single,language=bash]  
pip install python-dateutil
pip install pyparsing 
brew install matplotlib
\end{lstlisting}
\item Download and install PIL by typing into the terminal
\begin{lstlisting}[frame=single,language=bash]  
brew install Homebrew/python/pillow
\end{lstlisting}
\item Download and install Gmsh by typing into the terminal
\begin{lstlisting}[frame=single,language=bash]  
brew install gmsh
\end{lstlisting}
\item Download and install pyparse by typing into the terminal
\begin{lstlisting}[frame=single,language=bash]  
pip install pyparse
\end{lstlisting}
\item Download and install fipy by typing into the terminal
\begin{lstlisting}[frame=single,language=bash]  
pip install fipy
\end{lstlisting}
\item For a better performance during the FRAP simulations we recommend using the pysparse package. Currently it seems that this package is corrupted in pip, so building it from source is necessary.
To do so, go to \url{http://sourceforge.net/projects/pysparse/} and download the latest version of pysparse. Unpack the downloaded file and go to the folder containing the unpacked files by typing
\begin{lstlisting}[frame=single,language=bash]  
cd path/to/unpacked files
\end{lstlisting}
Build the pyparse package by typing 
\begin{lstlisting}[frame=single,language=bash]  
python setup.py install
\end{lstlisting}
\item Go to your PyFRAP folder by typing into the terminal
\begin{lstlisting}[frame=single,language=bash]  
cd path/to/PyFRAP/
\end{lstlisting}
and launch pyFRAP by typing into the terminal
\begin{lstlisting}[frame=single,language=bash]  
python pyfrp_app.py
\end{lstlisting}
If PyFRAP does not launch, open a Python terminal and try to import all necessary packages by typing into the terminal:
\begin{lstlisting}[frame=single,language=Python]  
import numpy
import scipy
import matplotlib
import matplotlib.image
import PyQt4
import code
import fipy
\end{lstlisting}
If you receive an error message while importing any of these
modules, try to reinstall the package or visit the development website of the problematic
package.


\end{itemize}

\subsection{Manual installation under Microsoft Windows}

Here we explain how to  manually  install and start PyFDAP on Microsoft Windows. The following instructions have only been tested for Microsoft Windows 8 (64-bit) and may differ for other versions.

\begin{itemize}
 \item Download and install the current version of Python 2.7x from \url{https://www.python.org/download/releases/.}
 \item Download and install the current version of PyQt4 from \url{http://www.riverbankcomputing.co.uk/software/pyqt/download}. The Windows installer will also install the required package SIP an
all necessary QT libraries.
 \item Download and install the current version of scipy-stack from \url{http://www.lfd.uci.edu/~gohlke/pythonlibs/#scipy-stack }. scipy-stack includes important Python packages such as nose, numpy, scipy
 and matplotlib. If you want to install the packages separately because there is no suitable installation binary of scipy-stack available, you can use the following links. However, we recommend using scipy-stack:
 \begin{itemize}
 \item Numpy: \url{http://sourceforge.net/projects/numpy/files/NumPy/} if you are running a 32bit system, on a 64 bit system go to \url{http://www.kfd.uci.edu/~gohlke/pythonlibs/}
 \item Scipy: \url{http://sourceforge.net/projects/scipy/files/scipy/}
 \item Matplotlib: \url{http://matplotlib.org/downloads.html}
 \item Nose: \url{https://nose.readthedocs.org/en/latest/}
 \item Ipython: \url{https://github.com/ipython/ipython/releases}
 \end{itemize}
 All packages on \url{http://www.lfd.uci.edu/~gohlke/pythonlibs} are provided in the wheel (.whl) format. An instruction on how to install wheel packages can be found in Section \ref{sec:wheel}.
\item  Download and install the current version of scikit-image from \url{http://www.lfd.uci.edu/~gohlke/pythonlibs/#scikit-image}.
\item  Download and install the current version of pyparse from \url{http://www.lfd.uci.edu/~gohlke/pythonlibs/#pysparse}. More details about pysparse can be found in Section \ref{sec:pysparsewin}.
\item  Download the current version of Gmsh from \url{http://geuz.org/gmsh/}. Unpack the zip file and copy it to your \textit{Programs} folder.
\item  Download and unpack the current version of PyFRAP from \url{http://people.tuebingen.mpg.de/mueller-lab/}.
\item Go to your PyFRAP folder and launch \verb+pyfrp_app.py+. If PyFRAP does not launch, open a Python terminal and try to import all necessary packages by typing:
\begin{lstlisting}[frame=single,language=Python]  
import numpy
import scipy
import matplotlib
import matplotlib.image
import PyQt4
import code
\end{lstlisting}

This should run without any problems. If you receive an error message while importing any of these modules, please try to reinstall the package or visit the development website of the problematic package.


\item NOTE: The PyFRAP video export has not been tested under Microsoft Windows. The required software Mencoder can be downloaded from \url{http://mplayerwin.sourceforge.net/downloads.html}.
\end{itemize}

\subsection{Adding a program to the PATH environment variable in Microsoft Windows}
\label{sec:pathosx}
The \verb+PATH+ environment variable is a variable containing folders where the Mac OSX's terminal looks for executables. Here we explain how to add a folder to this list such that entering the path
to the executable is not necessary anymore. 

\begin{itemize}
 \item Start the Terminal. The Terminal can be found in \textit{Applications $\rightarrow$ Utilities $\rightarrow$ Terminal}.
 \item To test if the program is already accesible through the \verb+PATH+, type
 \begin{lstlisting}[frame=single,language=bash]  
name_of_program
  \end{lstlisting}
  \item If you receive the error message \verb+name_of_program: command not found+, the program needs to be added to you \verb+PATH+.
  \item To add \verb+program+ to \verb+PATH+, edit .bash\_profile by launching the text editor \verb+nano+: 
\begin{lstlisting}[frame=single,language=bash]  
nano ~/.bash_profile
\end{lstlisting}
and add the following lines:
\begin{lstlisting}[frame=single,language=bash]  
export PATH="path/to/executable:$PATH"
\end{lstlisting}
Press \verb-Ctrl+O- and \verb-Ctrl+X- to save the new .bash\_profile and exit. 
\item After adding \verb+program+ to the \verb+PATH+ environment variable, restart the Terminal such that the Terminal updates \verb+PATH+. 
\end{itemize}

\subsection{Adding a program to the PATH environment variable in Microsoft Windows}
\label{sec:pathwin}
The \verb+PATH+ environment variable is a variable containing folders where the Microsoft Windows command prompt looks for executables. Here we explain how to add a folder to this list such that entering the path
to the executable is not necessary anymore. 

\begin{itemize}
 \item Start the command promt by pressing \verb#Win+R#, type  \verb+cmd+ and hit \verb+Enter+.
 \item To test if the program is already accesible through the \verb+PATH+, type
 \begin{lstlisting}[frame=single,language=bash]  
name_of_program
  \end{lstlisting}
  \item If you receive the error message \verb+'name_of_program' is not recognized as an internal or external command+, the program needs to be added to you \verb+PATH+.
  \item To add \verb+program+ to \verb+PATH+
  \begin{lstlisting}[frame=single,language=bash]  
set PATH=%PATH%;path/to/program
  \end{lstlisting}
 \item After adding \verb+program+ to the \verb+PATH+ environment variable, restart the command prompt such that the command prompt updates \verb+PATH+. 
\end{itemize}

\subsection{Installing wheel packages in Microsoft Windows}
\label{sec:wheel}
Wheel packages are a common way to distribute system specific precompiled python packages. You can install wheel packages using pip. 
\begin{itemize}
\item Start the command promt by pressing \verb#Win+R#, type  \verb+cmd+ and hit \verb+Enter+.
\item  Make sure that pip was added to your systems \verb+PATH+ by typing
 \begin{lstlisting}[frame=single,language=bash]  
  pip
 \end{lstlisting}
If this is not the case, follow the instructions in Section \ref{sec:pathwin} to add pip to the command prompt's \verb+PATH+ environment variable. Note that for standard Python installations, the path to pip should be \verb+C:\Python27\Scripts+.
\item To install wheel packages, enter into the command prompt
\begin{lstlisting}[frame=single,language=bash]  
cd path\to\wheelpackage
pip install wheelpackage.whl
  \end{lstlisting}
\end{itemize}

\subsection{Installing pysparse in Microsoft Windows}
\label{sec:pysparsewin}
Pysparse is a python package for handling sparse matrices. Having pysparse installed will substantially speed up FRAP simulations in PyFRAP. However, Anaconda does not provide pysparse and the it seems that currently  this package is corrupted in pip, so building it from a wheel is necessary.
To do so,
\begin{itemize}
\item Download  the current version of pyparse from \url{http://www.lfd.uci.edu/~gohlke/pythonlibs/#pysparse}.
\item Follow the instructions in Section \ref{sec:wheel} to install the downloaded wheel package.
\item In case of a 64bit system, there is a mistake in the file pysparse package that is easy to fix. The code of the function \verb+addAt+ in the file \verb+pysparseMatrix.py+ needs to be changes to:
\begin{lstlisting}[frame=single,language=python]  
 def addAt(self, vector, id1, id2):
    import numpy as np
    id1=id1.astype(np.int32)
    id2=id2.astype(np.int32)
    
    self.matrix.update_add_at(vector, id1, id2) 
  \end{lstlisting}
  If you use the Anaconda Python distribution according to Section \ref{sec:anaconda}, you should find this file in \\ 
  \verb+C:\Users\your_user_name\Anaconda\Lib\site-packages\fipy\matrices\pysparseMatrix.py+.\\
  For a standard Python installation, this file should be in \\ \verb+C:\Python27\Lib\site-packages\fipy\matrices\pysparseMatrix.py+.
  
\end{itemize}







\end{document}
